%%%%%%%%%%%%%%%%%%%%%%%%%%%%%%%%%%%%%%%
% Deedy - One Page Two Column Resume
% LaTeX Template
% Version 1.2 (16/9/2014)
%
% Original author:
% Debarghya Das (http://debarghyadas.com)
%
% Original repository:
% https://github.com/deedydas/Deedy-Resume
%
% IMPORTANT: THIS TEMPLATE NEEDS TO BE COMPILED WITH XeLaTeX
%
% This template uses several fonts not included with Windows/Linux by
% default. If you get compilation errors saying a font is missing, find the line
% on which the font is used and either change it to a font included with your
% operating system or comment the line out to use the default font.
% 
%%%%%%%%%%%%%%%%%%%%%%%%%%%%%%%%%%%%%%
% 
% TODO:
% 1. Integrate biber/bibtex for article citation under publications.
% 2. Figure out a smoother way for the document to flow onto the next page.
% 3. Add styling information for a "Projects/Hacks" section.
% 4. Add location/address information
% 5. Merge OpenFont and MacFonts as a single sty with options.
% 
%%%%%%%%%%%%%%%%%%%%%%%%%%%%%%%%%%%%%%
%
% CHANGELOG:
% v1.1:
% 1. Fixed several compilation bugs with \renewcommand
% 2. Got Open-source fonts (Windows/Linux support)
% 3. Added Last Updated
% 4. Move Title styling into .sty
% 5. Commented .sty file.
%
%%%%%%%%%%%%%%%%%%%%%%%%%%%%%%%%%%%%%%%
%
% Known Issues:
% 1. Overflows onto second page if any column's contents are more than the
% vertical limit
% 2. Hacky space on the first bullet point on the second column.
%
%%%%%%%%%%%%%%%%%%%%%%%%%%%%%%%%%%%%%%


\documentclass[]{deedy-resume-openfont}
\usepackage{fancyhdr}
 
\pagestyle{fancy}
\fancyhf{}
% \usepackage{verbatim}
 
\begin{document}

%%%%%%%%%%%%%%%%%%%%%%%%%%%%%%%%%%%%%%
%
%	 LAST UPDATED DATE
%
%%%%%%%%%%%%%%%%%%%%%%%%%%%%%%%%%%%%%%
\lastupdated

%%%%%%%%%%%%%%%%%%%%%%%%%%%%%%%%%%%%%%
%
%	 TITLE NAME
%
%%%%%%%%%%%%%%%%%%%%%%%%%%%%%%%%%%%%%%
\namesection{Thibaut}{Montes}{
\href{https://montest.github.io/}{montest.github.io} | \href{mailto:montest@lpsm.paris}{montest@lpsm.paris}
}

%%%%%%%%%%%%%%%%%%%%%%%%%%%%%%%%%%%%%%
%
%	 COLUMN ONE
%
%%%%%%%%%%%%%%%%%%%%%%%%%%%%%%%%%%%%%%

% \begin{minipage}[t]{0.33\textwidth} 
\begin{minipage}[t]{0.33\textwidth}


%%%%%%%%%%%%%%%%%%%%%%%%%%%%%%%%%%%%%%
%	 RESEARCH PROJECTS
%%%%%%%%%%%%%%%%%%%%%%%%%%%%%%%%%%%%%%

\section{Research Projects}

\subsection{Optimal Quantization}
\vspace{\topsep}
\begin{tightemize}
\item New \textbf{Weak Error} bounds and expansions for \textbf{Optimal Quantization} (Paper Submitted: \href{https://arxiv.org/abs/1903.10330}{\textbf{arXiv link}}).
\item Introducing a Randomized Heston Model with pricing solutions: Hybrid quantization tree using \textbf{Product Recursive Quantization} (Paper in progress).
\item Optimize existing methods in order to build optimal quantizers: \textbf{Fixed Point Research Acceleration}. 
\end{tightemize}
\sectionsep
% \vspace{\topsep}
\subsection{Multilevel Monte-Carlo}
\vspace{\topsep}
\begin{tightemize}
\item Optimizing xVA's risk (counterparty risk) computation using Multilevel Monte-Carlo that allows us to \textbf{\textit{kill the bias}} while \textbf{reducing the variance} of the estimator. 
\end{tightemize}




%%%%%%%%%%%%%%%%%%%%%%%%%%%%%%%%%%%%%%
%	 SKILLS
%%%%%%%%%%%%%%%%%%%%%%%%%%%%%%%%%%%%%%

\vspace{\topsep}
\vspace{\topsep}
\section{Links} 
Website:// \href{https://montest.github.io}{\bf montest.github.io} \\
Github:// \href{https://github.com/montest}{\bf montest} \\
LinkedIn:// \href{https://www.linkedin.com/in/thibaut-montes-194a77a9}{\bf thibaut-montes-194a77a9}

%%%%%%%%%%%%%%%%%%%%%%%%%%%%%%%%%%%%%%
%	 SKILLS
%%%%%%%%%%%%%%%%%%%%%%%%%%%%%%%%%%%%%%

\vspace{\topsep}
\vspace{\topsep}
\section{Skills}

\subsection{Programming}
\vspace{\topsep}
\begin{tightemize}
\item \location{Over 5000 lines:}
C++ $\diamond$ \LaTeX \\
\item \location{Over 1000 lines:}
Scala $\diamond$ Python $\diamond$ Matlab \\
\item \location{Discovering:}
Kafka $\diamond$ MongoDB
\end{tightemize}
% \location{Over 5000 lines:}
% C++ \textbullet{} \LaTeX \\ 
% \location{Over 1000 lines:}
% Scala \textbullet{} Python \textbullet{} Matlab \\
% \location{Discovering:}
% Kafka \textbullet{} MongoDB

\sectionsep
\subsection{Languages}
\vspace{\topsep}
\begin{tightemize}
\item French: native
\item English: fluent
\end{tightemize}

%%%%%%%%%%%%%%%%%%%%%%%%%%%%%%%%%%%%%%
%	 Interests
%%%%%%%%%%%%%%%%%%%%%%%%%%%%%%%%%%%%%%

\vspace{\topsep}
\vspace{\topsep}
\section{Interests}
Running $\diamond$ Trails

%%%%%%%%%%%%%%%%%%%%%%%%%%%%%%%%%%%%%%
%
%	 COLUMN TWO
%
%%%%%%%%%%%%%%%%%%%%%%%%%%%%%%%%%%%%%%

\end{minipage} 
\hfill
\begin{minipage}[t]{0.64\textwidth} 


%%%%%%%%%%%%%%%%%%%%%%%%%%%%%%%%%%%%%%
%	 EDUCATION
%%%%%%%%%%%%%%%%%%%%%%%%%%%%%%%%%%%%%%

\section{Education}
% \vspace{\topsep}

\runsubsection{PhD CIFRE in Numerical Probability}
\descript{| Laboratoire de Probabilités, Statistiques et Modélisation (LPSM) | Pierre et Marie Curie University (Paris VI)}
\location{Mar 2017 – Mar 2020 (expected) | Paris, France}
Under the direction of \textbf{Gilles Pagès} and \textbf{Vincent Lemaire} at the LPSM and the supervision of \textbf{Jean-Michel Fayolle} at the Independent Calculation Agent, a Fintech whose aim is to efficiently compute risk measures linked to counterparty default. My research subjects are Optimal Quantization, also known as K-means, and Multilevel Monte-Carlo methods.
% \vspace{\topsep}
\sectionsep


\runsubsection{ Research Master in Probability and Finance (with honors)}
\descript{| Pierre et Marie Curie University (Paris VI) in collaboration with Ecole Polytechnique }
\location{Sep 2014 – Jun 2016 | Paris, France}
\vspace{\topsep}
\begin{tightemize}
	\item Numerical Probability (Monte-Carlo, Sensitivities Computation, $\dots$).
	\item Stochastic Algorithms (Stochastic Gradient Descent, $\dots$).
	\item Stochastic Calculus and Control.
	\item Machine Learning.
\end{tightemize}
\sectionsep


\runsubsection{Bachelor degree in Mathematics (with honors)}
\descript{| Aix-Marseille University}
\location{Sep 2011 – Jun 2014 | Paris, France}
I spent the 2013-2014 academic year on exchange with the ERASMUS program in the Lund University's mathematics department, Lund, Sweden.


%%%%%%%%%%%%%%%%%%%%%%%%%%%%%%%%%%%%%%
%	 EXPERIENCE
%%%%%%%%%%%%%%%%%%%%%%%%%%%%%%%%%%%%%%

\vspace{\topsep}
\vspace{\topsep}
\section{Professional Experience}

\runsubsection{Quantitative Analyst}
\descript{| The Independent Calculation Agent}
\location{ From Nov 2016 | Paris, France}
As PhD candidate in collaboration with The ICA, I worked on the following projects:
\begin{tightemize}
\item Optimizing the ICA's analytic library using Optimal Quantization based methods (Pricing of Exotic Options in the interest rate world).
\item Identifying and \textit{killing} bias in the xVA computation using Multilevel Monte-Carlo methods.
\item Implementing new regulatory risk measures in the library.
\end{tightemize}

\sectionsep

\runsubsection{Intern}
\descript{| The Independent Calculation Agent}
\location{May 2016 – Oct 2016 | Paris, France}
The goal of my internship was to optimize financial products price and risk measures sensitivities computations. I investigated both Malliavin calculus and finite differences methods and concluded the latter offered the best optimization.
\sectionsep

\runsubsection{Intern}
\descript{| LPSM (former LPMA), under the direction of Daphné Giorgi and Vincent Lemaire}
\location{Jun 2015 – Jul 2015 | Paris, France}
I explored short rate models (Vasicek Model) and their numerical simulation using trinomial trees. The results of the project can be accessed at the following link: \href{http://simulations.lpsm.paris/trinomial_trees/}{\bf Trinomial Trees Simulation}
\sectionsep

\end{minipage} 
\end{document}